\documentclass[10pt]{amsart} 
\usepackage{graphicx} 
\usepackage{float} % necessary for placement of figures
\usepackage{amsmath}
\usepackage{ upgreek }
\usepackage{array}
\usepackage[style = authoryear, sorting = nyt, backend = biber]{biblatex}
\addbibresource[location = local, type = file]{C:/Users/bloh356/Google Drive/Library/Library.bib}
% \addbibresource[location = local, type = file]{/Users/bloh356/Google Drive/Library/Library.bib}

\begin{document}
The load duration curve is constructed from hourly system production data, which is arranged in descending order of magnitude. 
The units are MW on the y-axis and hours on the x-axis. 
From the curve we can estimate the amount of time that demand was above various thresholds. 
Do they estimate a load duration curve and then 

The loss of load expectation (LOLE) is defined as the number of periods in which demand exceeds the available capacity.


\section{Literature}
Decentralized GEP approach, whereby each GENCO establishes its own multi-year generation expansion plan to maximize its profit (i.e., electricity price minus plant costs) through a mixed integer problem using genetic algorithm solution methods \parencite{pereira2011generation}. 
The global level of the model performs checks on whether global constraints are met (i.e., prevent over investment in a technology, reliability, etc.). 
If constraints are violated then a new set of individual plans is created whereby each GENCO use updated estimates of the fuel and electricity prices, as well as capacity factors that are modeled in a systems dynamic model. 
Changes to these factors would result in changes to individual GENCO profits and subsequently, investment decisions. 
If a proper price signal cannot be sent in the systems dynamic model then two alternatives are used: imposing minimum investment capacities or incorporate a capacity term in the objective function \parencite{pereira2011generation}. 
Some models modify the objective function to include energy not served at the value of lost load.

A centralized, multi-period, multi-objective generation expansion planning model that incorporates sustainable energy sources is developed \parencite{aghaei:2013aa}.  
Reliability is evaluated using the Z-method.
The Z-method calculates the surplus of available generation at each hour $i$ of time period $t$. 
The surplus is a random variable with 'resource adequacy expressed as the ratio of the expected surplus to the standard deviation of the surplus in the time period $t$ \parencite{aghaei:2013aa}.

\begin{equation}	
\begin{split}
S_{it} &= R_{it} - L_{it} \\
\bar{S}_{t} &= \bar{R}_{t} - \bar{L}_{t} \\
Z_{t} &= \frac{\bar{S}_{t}}{\sigma_{S_t}}
\end{split}
\end{equation} 

\begin{equation}	
\begin{split}
\bar{S}_{0} &= \bar{R}_{0} - \bar{L}_{0} \\
&= \sum_{n \in N_{exs}} [(1-q_n)\cdot X_{n0} \cdot p_{n0}] - \bar{L}_0 \\
\bar{S'}_{t} &= \bar{R'}_{t} - \Updelta{L}_{t} \\
&= \sum_{n \in N_{exs}} [(1-q_n)\cdot X_{nt} \cdot p_{nt}] - \Updelta{L}_t  \forall t \in T \\
\Updelta{L}_{t} &= \bar{L}_{t} - \bar{L}_{0} \forall t \in T \\
\bar{S}_{t} &= \bar{S}_{0} + \bar{S'}_{t} \forall t \in T \\
\sigma^2_{S_0} &= \sum_{n \in N_{exs}} [q_n \cdot (1-q_n)\cdot X_{n0} \cdot p^2_{n0}] \\
\sigma^2_{R'_t} &= \sum_{n \in N_{exs}} [q_n \cdot (1-q_n)\cdot X_{nt} \cdot p^2_{nt}] \\
\sigma_{S_t} &= \sqrt{(\sigma^2_{S_0} + \sigma^2_{L}) + \sigma^2_{R'_t}}
\end{split}
\end{equation}

The objective function (one of the several objective functions) then maximizes the following: 
\begin{equation}
max f_3 = \sum_{t \in T} Z_t
\end{equation}

The uses a stochastic generation expansion planning model to model the interdependent operations of the natural gas and electric power system \parencite{pantos:2013aa}.
All possible system states are generated using Monte Carlo simulation methods, which are then used to create scenarios of all possible states in the time horizon \parencite{pantos:2013aa}. 
The scenarios consider all possible states of both the natural gas pipeline infrastructure, as well as the forced outage rate of natural gas fired generators. 
The model uses the Benders decomposition to solve, whereby the master problem is to optimize investments in new generation and energy transmission paths and the subproblem addresses the reliability check.
The reliability check problem is further decomposed using Benders into  the master problem, which addresses the reliability of the electric power system and the subproblem, which is focused on the natural gas system reliability \parencite{pantos:2013aa}. 
\cite{pantos:2013aa} use the loss of energy probability (LOEP) metric, which is defined as the ratio of the EEUE to the total electric energy demand. 
The way that they have implemented the subproblem (i.e., electric power system reliability) they can use a benders cut to provide information back to the master problem if the reliability constraint is violated (i.e., investment problem). 
\begin{equation}
LOEP_{bt} = \frac{\sum_{s=1}^{NS} PR_{s}\cdot W^{r}_{bt, s}\cdot DT_{b}}{\sum_{s=1}^{NS} PR_{s}\cdot L_{bt, s}\cdot DT_{b}}
\end{equation}
If the reliability criterion is not satistifed then the Benders cut is:
\begin{equation}
\sum_{s=1}^{NS} PR_{s}\cdot W^{r}_{bt,s} + \sum_{s=1}^{NS}\sum_{i=1}^{CG}(PR_{s}\cdot\lambda^{r}_{ibt,s}\cdot PG^{C,max}_{i}\cdot UX^{C}_{ibt,s}\cdot (X_{it}-\hat{X}_{it}))\leq LOEP\cdot \sum_{s=1}^{NS}PR_{s}\cdot L_{bt,s}
\end{equation}

Cumulant method

\cite{tohidi2013generation} include both generation expansion and retirement considerations through a stochastic mixed integer program.
The authors use a scenario reduction approach that minimizes the number of scenarios needed while maximizing the estimated standard deviation of the loss of load expectation (LOLE). 
The stopping criteria is formulated as:
\begin{equation}
\sigma = \frac{1}{NS}\sqrt{\sum_{s=1}^{NS}\frac{LOLE_{s}-\bar{LOLE})^2}{NS-1}}
\end{equation}
The model calculates the system minutes as derived from the Expected Energy Not Served (EENS). 
\begin{equation}
\begin{split}
SM_{y} &= \frac{EENS_{y}}{\bar{L}_{y}}\cdot 60 \\
EENS_{y} &= \sum_{l=1}^{ND} t_{l} \sum_{s=1}^{NS_y} \pi_{ys} \sum_{b=1}^{NB} LS_{ybls} \\
SM_{y} &\leq \bar{SM}
\end{split}
\end{equation}
Violations of the constraint are included in the objective function through a system risk cost (i.e., the value of lost load).
\begin{equation}
RC_{y} = (1+r)^{(0.5-y)}\sum_{b=1}^{NB}\sum_{l=1}^{ND}\sum_{s=1}^{NS_y} \pi_{ys} \cdot VOLL_{yb} \cdot LS_{ybls}
\end{equation}

\cite{jirutitijaroen2008reliability} use a stochastic mixed-integer program approach to the transmission and generation expansion planning problem.
Reliability is incorporated through the expected cost of load loss, which is estimated using scenario reduction techniques.
[Need to keep reading this one] 

\printbibliography
\end{document}