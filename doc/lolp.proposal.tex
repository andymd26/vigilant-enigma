\documentclass[10pt]{amsart} 
\usepackage{graphicx} 
\usepackage{float} % necessary for placement of figures
\usepackage{amsmath}
\usepackage{ upgreek }
\usepackage{array}
\usepackage[style = authoryear, sorting = nyt, backend = biber]{biblatex}
\addbibresource[location = local, type = file]{C:/Users/bloh356/Google Drive/Library/Library.bib}
% \addbibresource[location = local, type = file]{/Users/bloh356/Google Drive/Library/Library.bib}

\begin{document}
The load duration curve is constructed from hourly system production data, which is arranged in descending order of magnitude. 
The units are MW on the y-axis and hours on the x-axis. 
From the curve we can estimate the amount of time that demand was above various thresholds. 
Do they estimate a load duration curve and then 

The loss of load expectation (LOLE) is defined as the number of periods in which demand exceeds the available capacity.


\section{Literature}
Decentralized GEP approach, whereby each GENCO establishes its own multi-year generation expansion plan to maximize its profit (i.e., electricity price minus plant costs) through a mixed integer problem using genetic algorithm solution methods \parencite{pereira2011generation}. 
The global level of the model performs checks on whether global constraints are met (i.e., prevent over investment in a technology, reliability, etc.). 
If constraints are violated then a new set of individual plans is created whereby each GENCO use updated estimates of the fuel and electricity prices, as well as capacity factors that are modeled in a systems dynamic model. 
Changes to these factors would result in changes to individual GENCO profits and subsequently, investment decisions. 
If a proper price signal cannot be sent in the systems dynamic model then two alternatives are used: imposing minimum investment capacities or incorporate a capacity term in the objective function \parencite{pereira2011generation}. 
Some models modify the objective function to include energy not served at the value of lost load.

A centralized, multi-period, multi-objective generation expansion planning model that incorporates sustainable energy sources is developed \parencite{aghaei:2013aa}.  
Reliability is evaluated using the Z-method.
The Z-method calculates the surplus of available generation at each hour $i$ of time period $t$. 
The surplus is a random variable with 'resource adequacy expressed as the ratio of the expected surplus to the standard deviation of the surplus in the time period $t$ \parencite{aghaei:2013aa}.

\begin{equation}	
\begin{split}
S_{it} &= R_{it} - L_{it} \\
\bar{S}_{t} &= \bar{R}_{t} - \bar{L}_{t} \\
Z_{t} &= \frac{\bar{S}_{t}}{\sigma_{S_t}}
\end{split}
\end{equation} 

\begin{equation}	
\begin{split}
\bar{S}_{0} &= \bar{R}_{0} - \bar{L}_{0} \\
&= \sum_{n \in N_{exs}} [(1-q_n)\cdot X_{n0} \cdot p_{n0}] - \bar{L}_0 \\
\bar{S'}_{t} &= \bar{R'}_{t} - \Updelta{L}_{t} \\
&= \sum_{n \in N_{exs}} [(1-q_n)\cdot X_{nt} \cdot p_{nt}] - \Updelta{L}_t  \forall t \in T \\
\Updelta{L}_{t} &= \bar{L}_{t} - \bar{L}_{0} \forall t \in T \\
\bar{S}_{t} &= \bar{S}_{0} + \bar{S'}_{t} \forall t \in T \\
\sigma^2_{S_0} &= \sum_{n \in N_{exs}} [q_n \cdot (1-q_n)\cdot X_{n0} \cdot p^2_{n0}] \\
\sigma^2_{R'_t} &= \sum_{n \in N_{exs}} [q_n \cdot (1-q_n)\cdot X_{nt} \cdot p^2_{nt}] \\
\sigma_{S_t} &= \sqrt{(\sigma^2_{S_0} + \sigma^2_{L}) + \sigma^2_{R'_t}}
\end{split}
\end{equation}

The objective function (one of the several objective functions) then maximizes the following: 
\begin{equation}
max f_3 = \sum_{t \in T} Z_t
\end{equation}

\printbibliography
\end{document}