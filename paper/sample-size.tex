\documentclass[10pt]{amsart} 
\usepackage{graphicx} 
\usepackage{float} % necessary for placement of figures
\usepackage{amsmath}
\usepackage[style = authoryear, sorting = nyt, backend = biber]{biblatex}

\begin{document}
Is this an appropriate use of the margin of error equation in order to calculate the necessary sample size?
\begin{equation}
ME = z_{score}*\sqrt{\frac{\hat{p}(1-\hat{p})}{n}}
\end{equation}
where,
\begin{flushleft}
ME is the desired margin of error \\
$z_{score}$ is the z-score for the desired confidence level \\
$\hat{p}$ is the best guess at the correct value of $p$ \\
$n$ is the sample size that we are calculating \\ 
\end{flushleft}

Then in our particular example, if we assume that the desired margin of error is 0.002 at the 95\% confidence level.
And we also assume that 5\% of the counts are in the tail, we would have the following equation for establishing the required sample size:
\begin{equation}
0.002 = 1.96*\sqrt{\frac{(0.05)(0.95)}{n}}
\end{equation}

The sample size required to achieve the desired margin of error around $\hat{p}$ is then 45,619.
Taking it one step further, we would be confident at the 95\% confidence level that the proportion of the total count found in the tail is between [0.0498, 0.0502].
 
\end{document}